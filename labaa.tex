\documentclass[a4paper, 14pt]{extarticle}

% Подключение пакетов
\usepackage{fontspec}
\setmainfont[Ligatures=TeX]{Times New Roman}

% Настройка стиля документа
\tolerance=1
\emergencystretch=\maxdimen
\hyphenpenalty=10000
\hbadness=10000

% Дополнительные пакеты
\usepackage{amsmath} % для математических формул
\usepackage{listings} % для отображения кода
\usepackage{xcolor} % для цветной подсветки кода
\usepackage{geometry} % для изменения полей документа
\usepackage[utf8]{inputenc} % для поддержки кириллицы
\usepackage[russian]{babel} % для русского языка
\usepackage{graphicx} % для вставки изображений
\usepackage{hyperref} % для создания активных ссылок

% Настройка полей
\geometry{top=17mm, bottom=20mm, left=20mm, right=20mm}

% Настройка отображения кода
\lstset{
	language=Python,
	basicstyle=\ttfamily\small,
	keywordstyle=\color{blue}\bfseries,
	stringstyle=\color{red},
	commentstyle=\color{gray},
	showstringspaces=false,
	numbers=left,
	numberstyle=\tiny,
	stepnumber=1,
	numbersep=5pt,
	tabsize=4,
	breaklines=true,
	breakatwhitespace=false,
	showtabs=false,
	showspaces=false,
	showlines=true,
	frame=single,
	captionpos=b
}

\begin{document}
	
	% Титульный лист
	\begin{titlepage}
		\centering
		\large
		\textbf{Министерство науки и высшего образования Российской Федерации}\\
		\vspace{0.7cm}
		\textbf{Федеральное государственное автономное образовательное учреждение высшего образования}\\
		\textbf{«Национальный исследовательский университет ИТМО»}\\
		\vspace{1.5cm}
		\includegraphics[width=0.4\textwidth]{itmologo.jpg}\\
		\vspace{2cm}
		Факультет информационных технологий и программирования\\
		\vspace{3cm}
		{\normalsize Лабораторная работа №3}\\
		\vspace{0.5cm}
		{\LARGE \textbf{Работа с Latex}}\\
		\vfill
		\hfill
		\begin{flushright}
			\small
			\textbf{Выполнил студент группы № M3112} \\ 
			Плотников Александр Вадимович \\[1cm]
			
			\begin{minipage}{0.4\textwidth}
				\flushright
				\textbf{Проверил:} \\
				Хасан Карим Асадович / Жуков Артём Сергеевич\\
			\end{minipage}
		\end{flushright}
		
		\vfill
		Санкт-Петербург\\
		2024
	\end{titlepage}
	
	\newpage
	
	% Оглавление
	\tableofcontents
	
	\newpage
	
	\section{Задачи программ из репозитория}
	Программы из данного репозитория решают задачи вычисления площади и периметра различных геометрических фигур, таких как треугольники, квадраты и круги. Основная цель — предоставить пользователю возможность вычислять эти параметры с помощью минимального ввода данных.
	
	\section{Описание файлов программ из репозитория}
	Для каждого файла приведен исходный код программы, текстовое описание логики программы и визуализация используемых формул.
	
	\subsection{Файл \texttt{triangle.py}}
	\texttt{triangle.py} предоставляет функции для работы с треугольниками: вычисление площади и периметра.
	
	- Площадь треугольника:
	\[
	p = \frac{a + b + c}{2}
	\]
	где \( a \), \( b \), \( c \) — длины сторон треугольника.
	
	- Периметр треугольника:
	\[
	P = a + b + c
	\]
	
	\begin{lstlisting}[language=Python]
		def area(a, b, c):
			return (a + b + c) / 2
		
		def perimeter(a, b, c):
			return a + b + c
	\end{lstlisting}
	
	\subsection{Файл \texttt{square.py}}
	\texttt{square.py} содержит функции для вычисления площади и периметра квадрата.
	
	- Площадь квадрата:
	\[ 
	S = a^2 
	\]
	где \( a \) — длина стороны квадрата.
	
	- Периметр квадрата:
	\[ 
	P = 4a 
	\]
	
	\begin{lstlisting}[language=Python]
		def area(a):
			return a * a
		
		def perimeter(a):
			return 4 * a
	\end{lstlisting}
	
	\subsection{Файл \texttt{circle.py}}
	В файле \texttt{circle.py} представлены функции для работы с кругами: вычисление площади и периметра (длины окружности).
	
	- Площадь круга:
	\[ 
	S = \pi r^2 
	\]
	где \( r \) — радиус круга.
	
	- Периметр (окружность) круга:
	\[ 
	P = 2 \pi r 
	\]
	
	\begin{lstlisting}[language=Python]
		import math
		
		def area(r):
			return math.pi * r * r
		
		def perimeter(r):
			return 2 * math.pi * r
	\end{lstlisting}
	
	\newpage
	\subsection{Файл \texttt{calculate.py}}
	\texttt{calculate.py} — это основная программа, которая взаимодействует с пользователем и позволяет вычислять площадь и периметр выбранной фигуры.
	\begin{lstlisting}[language=Python]
		import circle
		import square
		
		figs = ['circle', 'square']
		funcs = ['perimeter', 'area']
		
		def calc(fig, func, size):
			result = eval(f'{fig}.{func}(*{size})')
			print(f'{func} of {fig} is {result}')
		
		if __name__ == "__main__": 
			func = ''
			fig = ''
			size = list()
			
			while fig not in figs:
			fig = input(f"Enter figure name, available are {figs}:\n")
			
			while func not in funcs:
			func = input(f"Enter function name, available are {funcs}:\n")
			
			while len(size) != 1:
			size = list(map(int, input("Input figure sizes separated by space, 1 for circle and square\n").split(' ')))
			
			calc(fig, func, size)
	\end{lstlisting}
	
	\newpage
	\subsection{Формулы используемых фигур}
	В данной секции представлены основные формулы для вычисления площади и периметра фигур, использованных в программе.
	
	\begin{itemize}
		\item Треугольник:
		\[
		S = \sqrt{p(p - a)(p - b)(p - c)}, \quad P = a + b + c
		\]
		где \(p = \frac{a + b + c}{2}\) — полупериметр треугольника.
		
		\item Квадрат:
		\[
		S = a^2, \quad P = 4a
		\]
		
		\item Круг:
		\[
		S = \pi r^2, \quad P = 2\pi r
		\]
	\end{itemize}
	
	\newpage
	\section{Ссылка}
	Ссылка на код Latex:
	\href{https://github.com/Alexandr-prog34/Laba-ISRPO-3}
	
\end{document}